\begin{abstract}

合成孔径雷达(synthetic aperture radar, SAR)是一种微波雷达成像技术。SAR 借助通过机载或星载微波雷达观察地面反射信号,并将不同时间观测的信号组合、得到尺度远大于雷达口径的“合成孔径”雷达数据。SAR 通过数字信号处理将微波信号光学分辨率从千米量级提升至米级。干涉合成孔径雷达(Interferometric SAR, InSAR)则使用两幅或更多的 SAR 图像计算反射波相位差,得到高精度的地面高程信息或者形变信息。通过 InSAR 技术可以取得数日到数年时间尺度上厘米级的地表形变信息,是地球物理研究中常用的大地测量数据源。

SAR/InSAR 成像数据处理量非常大,算法时间和空间复杂度较高。为更加即时地处理数据和进行研究,设计高效而精确的 SAR/InSAR 数据处理算法非常有必要。自 SAR 技术出现以来,已经有许多 SAR/InSAR 算法优化方面的工作,其中一部分工作通过新的算法降低了计算复杂度,而另一部分算法则致力于在现有计算设备(如多核 CPU、GPU 和 FPGA)上高效地实现 SAR/InSAR 数据处理。

本工作基于开源 InSAR 数据处理软件 GMTSAR 公开的算法,选取图像配准(image alignment)等计算复杂度高的程序模块,提出了利用多 CPU 并行计算进行高效 InSAR 数据处理的方法。我们在公开的 ALOS 卫星数据上测试了并行优化的 InSAR 成像程序,证明 CPU 并行算法可以在不降低结果精度的条件下显著提高桌面工作站和服务器平台上 InSAR 数据处理的效率。

\keywords{合成孔径干涉雷达\zhspace{} 大地测量\zhspace{} 图像处理\zhspace{} 并行计算}
\end{abstract}

%\begin{enabstract}
%This is a sample document of USTC thesis \LaTeX{} template for bachelor, master
%and doctor. The template is created by zepinglee and seisman, which orignate from
%the template created by ywg@USTC\@. The template meets the equirements of USTC
%theiss writing standards.
%
%This document will show the usage of basic commands provided by \LaTeX{} and some
%features provided by the template. For more information, please refer to the
%template document ustcthesis.pdf.
%
%\enkeywords{University of Science and Technology of China (USTC), Thesis, Universal \LaTeX{} Template, Bachelor, Master, PhD}
%\end{enabstract}
