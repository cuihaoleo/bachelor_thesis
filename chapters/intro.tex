\chapter{绪论}


\section{研究背景}

合成孔径雷达(synthetic aperture radar, SAR)是一种高分辨率的微波雷达成像技术。SAR 通过机载或星载微波雷达发射相干的微波信号并观察地面回波,通过信号处理得到尺度远大于雷达口径的等效“合成孔径”。SAR 成像分辨率可以达到米级,部分机载 SAR 系统甚至可以提供 10 cm 级的地表分辨率。由于 SAR 的高分辨率和全天时、全天候工作的优势,SAR 成像被广泛地用于灾害预警、遥感测绘、海洋观测、环境保护等军用和民用领域。自上世纪 80 年代以来,各国陆续发射了 ERS-1/2、Envisat、ALOS 等 SAR 卫星,星载 SAR 雷达数据成为重要的大地测量数据来源。

干涉合成孔径雷达(Interferometric SAR, InSAR)是利用多幅 SAR 图像提取高程信息的拓展应用。如果利用一组 SAR 雷达观测同一块地面区域、或者同一雷达不同时间观测同一块地面区域,由于两次观测中天线与目标相对位置存在差异(由于天线位置差异或地表形变),两幅 SAR 图像之间就会存在相位差异。相位差反映了地表高程信息或者形变信息。InSAR 技术可以取得数日到数年时间尺度上厘米级的地表形变信息,被广泛地用于地震、地面沉降、固体潮等长短周期的地表形变观测。

SAR/InSAR 成像算法是一个复杂的信号处理过程。典型的 SAR 成像算法如距离多普勒算法,它包括距离向压缩(range compression)、距离向徙动校正(range migration)和方位向压缩(azimuth compressions)三个基本步骤。信号压缩通过匹配滤波实现,可以借助快速傅立叶变换算法在频域高效实现。SAR 数据处理技术已经比较成熟,可以通过数字信号处理器高效地实现。包括日本 ALOS 在内的许多 SAR 数据源直接向终端用户提供了 SAR 成像处理后的单视复图像(single look complex,SLC)。因此,对于地球物理研究工作等终端应用,往往不需要接触 SAR 成像算法,直接对获取的 SLC 图像进行 InSAR 成像即可。

InSAR 成像算法时间和空间复杂度比较高。在大地测量领域,InSAR 成像范围宽度可达数百公里,成像数据量十分巨大,这对成像程序的性能提出了很高要求。科研工作者通常使用个人电脑或服务器 CPU 进行 InSAR 数据处理,这类桌面平台上常用的 SAR/InSAR 数据处理软件有 GMTSAR、ISCE(InSAR Scientific Computing Environment)等。

近年来,桌面中央处理器(Central Processing Unit, CPU)主频受制于物理极限已趋于饱和,厂商开始发展多核心 CPU 技术,提高 CPU 并行处理能力。近几年的消费级个人电脑往往都已经配备具有 4 到 8 个中央处理单元的多核心 CPU;服务器甚至可能配置多个多核心 CPU,达到数十个逻辑计算核心。但无论是个人电脑还是服务器,乃至具有312万个计算核心的天河二号超级计算机,其单任务处理能力仍是十分有限(CPU 主频一般在 1GHz 到 4GHz)。只有运行多个计算任务、或者运行专门设计的并行处理程序,才能充分发挥多核心的并行优势。

以 GMTSAR 为代表的传统串行 InSAR 处理软件无法发挥多核心 CPU 的并行处理能力,计算能力仍受制于单核心计算性能。为了满足大地测量中大规模 InSAR 数据处理的需求,本课题对 GMTSAR 的 InSAR 处理算法的部分重要模块进行了并行优化,以期在桌面多核心 CPU 平台上减少数据处理时间。

\section{现有工作}

现在,InSAR 成像算法已经比较成熟,主要包括图像配准、干涉相位合成、相位滤波和相位解缠等基本模块。加州大学圣迭戈分校(UC San Diego)和斯克里普斯海洋研究所(Scripps Institution of Oceanography)开发的 SAR/InSAR 处理软件 GMTSAR 是学习 InSAR 成像算法的重要范本。GMTSAR 仅能使用 CPU 单线程进行 InSAR 成像,计算效率不高。但由于其源代码完全开放,任何人都可以免费获取并自由修改,因此在地球物理学术界应用非常广泛。

目前已经有一些针对特定硬件优化 InSAR 算法的实例。\citet{shayu2014} 在多核数字信号处理器(Digital Signal Process, DSP)上实现了实时 InSAR 数据处理,并探讨了在现场可编程逻辑门阵列(Field Programmable Gate Array, FPGA)硬件上实现高效并行 InSAR 数据处理的可行性。这类多核 DSP 芯片和 FPGA 设备具有低功耗和良好的并行性,适合在嵌入式设备上使用,比如可以整合进 SAR 雷达系统。但在桌面平台,这类特殊硬件非常少见。

图形处理器(Graphics Processing Unit, GPU)是另一类具有强大并行处理能力的微处理器,并且往往具有较大的独立内存空间(显存)。\citet{reza2015accelerating} 在 NVIDIA GPU 上借助 CUDA 通用计算框架实现了高效的相位解缠算法,对比 CPU 程序取得了数十倍的性能提升,并显著降低了计算功耗。未来,基于 GPU 的通用计算有望成为高性能计算应用的主要平台。但目前 GPU 通用计算技术还在发展阶段,在科学计算领域应用并不广泛。

\section{课题内容简介}

本课题基于 GMTSAR 公开的 InSAR 成像算法,选取图像配准(image alignment)和图像拼接两个计算复杂度比较高的程序模块,设计并实现了多 CPU 并行处理算法,以期在桌面或服务器多核心 CPU 上改善 InSAR 成像程序的性能。本文详细介绍了两个并行处理算法的设计方案,并简要探讨了其他 InSAR 数据处理模块的并行优化思路。

在公开的 ALOS 卫星数据上进行的测试显示,CPU 并行处理算法在不降低结果精度的条件下显著提高了多核心 CPU 平台上 InSAR 数据处理的效率。本文给出了并行算法与 GMTSAR 串行算法的性能比较数据,并定性地分析了性能提升的来源。

