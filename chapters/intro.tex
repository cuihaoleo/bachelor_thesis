\chapter{绪论}


\section{研究背景}

合成孔径雷达(synthetic aperture radar, SAR)是一种高分辨率的微波雷达成像技术。SAR 通过机载或星载微波雷达发射相干的微波信号并观察地面回波,通过信号处理得到尺度远大于雷达口径的等效“合成孔径”。SAR 成像分辨率可以达到米级,部分机载 SAR 系统甚至可以提供 10 cm 级的地表分辨率。由于 SAR 的高分辨率和全天时、全天候工作的优势,SAR 成像被广泛地用于灾害预警、遥感测绘、海洋观测、环境保护等军用和民用领域。自上世纪 80 年代以来,各国陆续发射了 ERS-1/2、Envisat、ALOS 等 SAR 卫星,星载 SAR 雷达数据成为重要的大地测量数据来源。

干涉合成孔径雷达(Interferometric SAR, InSAR)是利用多幅 SAR 图像提取高程信息的拓展应用。如果利用一组 SAR 雷达观测同一块地面区域、或者同一雷达不同时间观测同一块地面区域,由于两次观测中天线与目标相对位置存在差异(由于天线位置差异或地表形变),两幅 SAR 图像之间就会存在相位差异。相位差反映了地表高程信息或者形变信息。InSAR 技术可以取得数日到数年时间尺度上厘米级的地表形变信息,被广泛地用于地震、地面沉降、固体潮等长短周期的地表形变观测。

SAR/InSAR 成像算法是一个复杂的信号处理过程。典型的 SAR 成像算法如距离多普勒算法,它包括距离向压缩(range compression)、距离向徙动校正(range migration)和方位向压缩(azimuth compressions)三个基本步骤。信号压缩通过匹配滤波实现,可以借助快速傅立叶变换算法在频域高效实现。SAR 数据处理技术已经比较成熟,可以通过数字信号处理器高效地实现。包括日本 ALOS 在内的许多 SAR 数据源直接向终端用户提供了 SAR 成像处理后的单视复图像(single look complex,SLC)。因此,对于地球物理研究工作等终端应用,往往不需要接触 SAR 成像算法,直接对获取的 SLC 图像进行 InSAR 成像即可。

InSAR 成像算法包括图像配准、干涉相位合成、相位滤波和相位解缠等基本模块,算法时间和空间复杂度比较高。在大地测量领域,InSAR 成像范围宽度可达数百公里,成像数据量十分巨大,这对成像程序的性能提出了很高要求。科研工作者通常使用个人电脑或服务器 CPU 进行 InSAR 数据处理,这类桌面平台上常用的 SAR/InSAR 数据处理软件有 GMTSAR、ISCE(InSAR Scientific Computing Environment)等。

近年来,桌面 CPU 主频受制于物理极限已趋于饱和,厂商开始发展多核心处理器技术,提高处理器并行处理能力。近几年的消费级个人电脑往往都已经配备具有 4 到 8 个中央处理单元的多核心处理器;服务器甚至可能配置多个多核心处理器,达到数十个逻辑计算核心。但无论是个人电脑还是服务器,乃至具有312万个计算核心的天河二号超级计算机,其单任务处理能力仍是十分有限(主频一般在 1GHz 到 4GHz)。只有运行多个计算任务、或者运行专门设计的并行处理程序,才能充分发挥多核心的并行优势。

以 GMTSAR 为代表的传统串行 InSAR 处理软件无法发挥多核心处理器的并行处理能力,计算能力仍受制于单核心计算性能。为了满足大地测量海量 InSAR 数据处理的需求,本课题对 GMTSAR 的 InSAR 处理算法的部分重要模块进行了并行优化。

\section{现有工作}

测试引用 \cite{shayu2014}。

本工作基于开源 InSAR 数据处理软件 GMTSAR 公开的算法,选取图像配准(image alignment)等计算复杂度高的程序模块,提出了利用多 CPU 并行计算进行高效 InSAR 数据处理的方法。我们在公开的 ALOS 卫星数据上测试了并行优化的 InSAR 成像程序,证明 CPU 并行算法可以在不降低结果精度的条件下显著提高桌面工作站和服务器平台上 InSAR 数据处理的效率。

\section{课题内容简介}


