\chapter{绪论}


\section{研究背景}

合成孔径雷达(synthetic aperture radar, SAR)是一种高分辨率的微波雷达成像技术。机载或星载 SAR 微波雷达发射相干的微波信号并观察地面回波,通过信号处理得到尺度远大于雷达口径的等效“合成孔径”。SAR 成像分辨率可以达到米级,部分机载 SAR 系统甚至可以提供 10 cm 级的地表分辨率\cite{cantalloube2006airborne}。由于其高分辨率和全天时、全天候工作的优势,SAR 成像被广泛地用于灾害预警、遥感测绘、海洋观测、环境保护等军用和民用领域。自上世纪 80 年代以来,各国陆续发射了 ERS-1/2、Envisat、ALOS 等 SAR 卫星,星载 SAR 雷达数据成为重要的大地测量数据来源。

合成孔径雷达干涉(Interferometric SAR, InSAR)是利用多幅 SAR 图像提取高程信息的拓展应用。如果利用一组 SAR 雷达观测同一块地面区域、或者同一雷达不同时间观测同一块地面区域,由于两次观测中天线与目标相对位置存在差异(由于天线位置差异或地表形变),两幅 SAR 图像之间就会存在相位差异。相位差反映了地表高程信息或者形变信息。InSAR 成像是常用的雷达测高技术,可以取得数日到数年时间尺度上厘米级的地表形变信息,被广泛地用于地震、地面沉降、固体潮等长短周期的地表形变观测。

SAR/InSAR 成像算法是一个复杂的信号处理过程。典型的 SAR 成像算法如距离多普勒算法,它包括距离向压缩(range compression)、距离向徙动校正(range migration)和方位向压缩(azimuth compressions)三个基本步骤。信号压缩通过匹配滤波实现,可以借助快速傅立叶变换算法(Fast Fourier Transform, FFT)高效完成。SAR 数据处理技术已经比较成熟,可以通过数字信号处理器高效地实现\cite{curlander2006sar}。包括日本 ALOS 在内的许多 SAR 数据源直接向终端用户提供了 SAR 成像处理后的单视复图像(single look complex,SLC)。因此,对于地球物理研究工作等终端应用,往往可以直接下载 SLC 图像进行 InSAR 成像。

InSAR 数据处理算法时间和空间复杂度比较高。在大地测量领域,InSAR 成像范围宽度可达数百公里,处理数据量十分巨大。虽然经过数十年的发展 InSAR 数据处理程序的效率已经提高了很多,但研究人员往往会在个人工作站上进行 InSAR 数据处理,仍然需要耗费较长的时间等待结果。在桌面平台上,常用的 SAR/InSAR 数据处理软件有 GMTSAR\cite{web:gmtsar}、ISCE(InSAR Scientific Computing Environment)\cite{web:isce}等。

近年来,计算机中央处理器(Central Processing Unit, CPU)主频受制于物理极限已趋于饱和,芯片厂商转而发展多核心 CPU 技术,提高 CPU 并行处理能力。多核心 CPU 可以在同一时间并行处理多个计算任务,但每个任务的计算速度仍受制于主频。近几年的消费级个人电脑往往都已经配备具有 4 到 8 个计算核心的多核心 CPU;服务器甚至可能配置多个多核心 CPU,达到数十个逻辑计算核心。利用多个 CPU 核心进行多线程并行计算,可以使总体计算能力突破主频瓶颈。

在新兴的计算机视觉、人工智能等领域,多线程并行处理已经被广泛用于加速计算程序。而以 InSAR 数据处理为代表的地球物理科学计算领域,很多经典计算程序并未进行并行优化,计算速度仍受制于 CPU 主频。为了满足大地测量中大规模 InSAR 数据处理的需求,加速桌面平台 InSAR 数据处理,本课题以 GMTSAR 开放的算法代码为基础对 InSAR 数据处理算法的部分重要模块进行了并行优化,以期在桌面多核心 CPU 平台上减少数据处理时间。

\section{现有工作}

InSAR 成像算法已经发展得非常成熟,主要包括图像配准、干涉相位合成、相位滤波和相位解缠等基本模块。加州大学圣迭戈分校(UC San Diego)和斯克里普斯海洋研究所(Scripps Institution of Oceanography)开发的 SAR/InSAR 数据处理软件 GMTSAR 是学习这些算法的重要范本。GMTSAR 仅能使用 CPU 单线程进行 InSAR 数据处理,计算效率不高。但由于其源代码完全开放,任何人都可以免费获取并自由修改,因此在地球物理学术界应用非常广泛。\cite{sandwell2011gmtsar}

目前已经有一些针对特定硬件优化 InSAR 算法的实例。\citet{shayu2014} 在多核数字信号处理器(Digital Signal Process, DSP)上实现了实时 InSAR 数据处理,并探讨了在现场可编程逻辑门阵列(Field Programmable Gate Array, FPGA)硬件上实现高效并行 InSAR 数据处理的可行性。这类多核 DSP 芯片和 FPGA 设备具有低功耗和良好的并行性,适合在嵌入式设备上使用,比如可以整合进 SAR 雷达系统。但在桌面平台,这类特殊硬件非常少见。

图形处理器(Graphics Processing Unit, GPU)是另一类具有强大并行处理能力的微处理器,并且往往具有较大的独立内存空间(显存)。\citet{reza2015accelerating} 在 NVIDIA GPU 上借助 CUDA 通用计算框架实现了高效的相位解缠算法,对比 CPU 程序取得了数十倍的性能提升,并显著降低了计算功耗。未来,基于 GPU 的通用计算有望成为高性能计算应用的主要平台。但目前 GPU 通用计算技术还在发展阶段,在科学计算领域应用并不广泛。

\section{课题内容简介}

本课题基于 GMTSAR 公开的 InSAR 数据处理算法,选取图像配准(image registration)和图像拼接(image stitching)两个计算复杂度比较高的程序模块,设计并实现了多线程并行处理算法,以期在桌面或服务器多核心 CPU 上提高 InSAR 数据处理效率。本文详细介绍了两个并行处理算法的设计方案,并简要探讨了其他 InSAR 数据处理模块的并行优化思路。

在公开的 ALOS 卫星数据上进行的测试显示,多线程并行处理算法在不降低结果精度的条件下显著提高了多核心 CPU 平台上 InSAR 数据处理的效率。本文给出了并行算法与 GMTSAR 串行算法的性能比较数据,并定性地分析了性能提升的来源。

