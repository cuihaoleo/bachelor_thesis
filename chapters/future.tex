\chapter{总结与展望}

\section{工作总结}

本课题对开源 InSAR 数据处理软件 GMTSAR 中的图像配准模块、图像拼接模块进行了并行优化,以提高桌面多核心 CPU 处理器上 InSAR 数据处理的效率。在真实 SAR 数据上进行的实验显示,并行程序在保证程序正确性的前提下大幅缩短了数据处理时间。

本课题基于 GMTSAR 图像配准模块 xcorr,设计了并行图像配准程序 xcorr2。基于硬件环境的定性分析,xcorr2 采用主-副多线程并行结构,使用一个主线程进行文件读取和线程管理,主线程以采样窗口为单位将配准任务分派给计算线程处理。此外,xcorr2 程序还对算法中部分 FFT 算法进行了简化,如使用实序列 FFT 算法代替复序列 FFT 算法。实验对比显示,xcorr2 在不影响配准精度的前提下通过多线程并行显著缩短了配准算法执行时间、提高了多核心 CPU 利用率。

本课题还设计了并行图像拼接程序。图像拼接程序利用并行优化中常用的规约技术,将拼接任务进行多线程分治并多层迭代拼接得到结果。在实验测试中,并行规约技术将图像拼接程序执行时间降低了一半以上。

受课题时间与个人能力限制,本课题仅完成 GMTSAR 软件中部分模块的优化工作。为了弥补这一不足,本文对干涉图生成、相位解缠等其他模块的并行优化进行了简单的讨论和分析,希望今后能够继续完成这些未尽的工作。

\section{未来展望}


