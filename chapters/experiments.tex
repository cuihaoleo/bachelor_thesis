\chapter{实验测试与分析}

\section{性能测试}

本节使用的 InSAR 实验数据基于 GMTSAR 官方提供的2010年4月下加利福尼亚州 $7.2 M_w$ 地震前后的 ALOS 卫星数据。主、副 SAR 图像分别拍摄于2009年12月17日和2010年5月4日。SAR 图像大小为一个标准的 ALOS 图像帧,方位向和距离向长度各为 27648 像素和 11304 像素。软件测试环境如表 \ref{tab:env} 所示。

\begin{table}[htbp]
\centering
\begin{tabular}{|l|l|l|}
\hline
    \multirow{3}{*}{CPU}                        & 型号     & Intel Xeon E5-2650 v4                 \\ \cline{2-3} 
                                                & 核心数   & $\times 12$                           \\ \cline{2-3} 
                                                & 主频     & 2.20 GHz                              \\ \hline
    \multicolumn{2}{|l|}{内存}                             & 32768 MiB                             \\ \hline
    \multicolumn{2}{|l|}{操作系统}                         & Ubuntu 16.04.2 LTS, Linux 4.4.0 内核  \\ \hline
    \multicolumn{1}{|c|}{\multirow{2}{*}{编译}} & 编译器   & gcc 5.4.0                             \\ \cline{2-3} 
    \multicolumn{1}{|c|}{}                      & 编译参数 & \texttt{-O2 -march=native}            \\ \hline
\end{tabular}
\caption{软件测试环境} \label{tab:env}
\end{table}

并行算法的性能并非总是随着线程数的增加而提升。当计算线程增加时,线程间通信的成本会增加,其他不可并行资源访问(如磁盘读写)也可能达到性能瓶颈。图 \ref{fig:exp_cores} 显示了默认计算参数下 xcorr2 性能随计算线程数的变化趋势。当计算线程从无增加至6个时,图 \ref{fig:exp_cores_a} 显示的计算时间缩短非常显著。但当计算线程数进一步增加时,运行时间并没有缩短。图 \ref{fig:exp_cores_b} 中的 CPU 核心使用率也反应了这种趋势,当计算线程数多于6个时,核心使用率少于计算线程数,说明程序已经无法充分利用所有的 CPU 核心。后面的实验中将选取8个计算线程,以反映该测试平台下 xcorr2 程序的最优性能。

\begin{figure}[htbp]
\centering
\subfloat[程序计算时间]{
    \label{fig:exp_cores_a}
    \begin{minipage}[t]{0.49\textwidth}
        \centering
        \resizebox {\textwidth} {!} {
            \begin{tikzpicture}
\begin{axis}[
    xlabel={计算线程数},
    ylabel={计算时间/s},
    xmin=0, xmax=12.2,
    ymin=0, ymax=12,
    xtick={0, 2, 4, 6, 8, 10, 12},
%    ytick={ },
    legend pos=north east,
    ymajorgrids=true,
    grid style=dashed,
]

\addplot[
    color=red,
    mark=square,
    style=solid,
    ]
    coordinates {
        (0, 11.200907922)
        (1, 10.521500231)
        (2, 5.431319614)
        (3, 3.757715043)
        (4, 2.918596810)
        (6, 2.146015373)
        (8, 2.052885075)
        (12, 2.181156634)
    };
    \legend{xcorr2} 
\end{axis}
\end{tikzpicture}

        }
    \end{minipage}
}
\subfloat[CPU 核心利用率]{
    \label{fig:exp_cores_b}
    \begin{minipage}[t]{0.49\textwidth}
        \centering
        \resizebox {\textwidth} {!} {
            \begin{tikzpicture}
\begin{axis}[
    xlabel={计算线程数},
    ylabel={占用 CPU 核心数},
    xmin=0, xmax=12.2,
    ymin=0, ymax=10,
    xtick={0, 2, 4, 6, 8, 10, 12},
%    ytick={ },
    legend pos=north west,
    ymajorgrids=true,
    grid style=dashed,
]
\addplot[
    color=red,
    mark=square,
    ]
    coordinates {
        (12,8.667)
        (8 ,7.296)
        (6 ,6.414)
        (4 ,4.438)
        (3 ,3.376)
        (2 ,2.304)
        (1 ,1.168)
        (0 ,1.000)
    };
    \legend{xcorr2}
\end{axis}
\end{tikzpicture}

        }
    \end{minipage}
}

\caption{使用不同数量计算线程对 xcorr2 配准性能的影响} \label{fig:exp_cores}
\note{xcorr2 运行时会启动一个主线程和若干计算线程,主线程不参与计算。计算线程数为0代表不启动计算线程,直接使用主线程串行计算。}
\end{figure}
 
图 \ref{fig:exp_ri}、图 \ref{fig:exp_nxy} 和图 \ref{fig:exp_xsys} 展示了不同配准参数对 GMTSAR xcorr 程序、串行 xcorr2 程序(即不启动计算线程,直接使用主线程计算)和多线程并行 xcorr2 程序(使用8个计算线程)计算性能的影响。注意 GMTSAR xcorr 虽然主要算法是串行计算的,但由于其调用的 GMT 函数库会启动额外的线程进行一些非计算操作,因此实测的 CPU 核心占用率会高于1。

对于评测数据,简要分析如下:
\begin{itemize}
    \item \textbf{距离向插值因子 $r_i$}(图\ref{fig:exp_ri})决定了距离向插值后采样窗口距离向像素数增加的倍数,主要是增加了一组 2D FFT 计算。GMTSAR xcorr 启用距离向插值后计算时间明显增加,但并不随着插值因子增大而进一步变化,这可能是由于 xcorr 中 2D FFT 是通过对每一行数据进行 1D FFT 实现的,每一轮 1D FFT 规模相对较小,计算时间主要受每一行的数组操作(复制、补零等)支配。而 xcorr2 直接使用 2D FFT 完成插值,运行时间受到 FFT 算法点数支配。
    \item \textbf{采样窗口数量 $n_x \times n_y$}(图\ref{fig:exp_nxy})决定了配准算法处理的数据总量。理论上计算时间与采样窗口数量成正比,测试数据也印证了这一点。
    \item \textbf{采样窗口宽度 $w$}(图\ref{fig:exp_xsys})的平方即为采样窗口的大小,主要影响 2D FFT 算法规模。2D FFT 复杂度应当正比于 $w^2 \log(w)$,测试数据基本上反映了这一趋势(由于 $\log(w)$ 变化较小,计算时间基本与 $w^2$ 成正比)。
\end{itemize}

\begin{figure}[htbp]
\centering
\subfloat[程序计算时间]{
    \label{fig:exp_ri_a}
    \begin{minipage}[t]{0.49\textwidth}
        \centering
        \resizebox {\textwidth} {!} {
            \begin{tikzpicture}
\begin{semilogyaxis}[
    xlabel={距离向插值因子},
    ylabel={计算时间/s},
    xmin=0, xmax=8,
    ymin=1, ymax=1000,
    xtick={0, 2, 4, 6, 8, 10, 12},
    ytick={1, 2, 5, 10, 20, 50, 100, 200, 500 },
    legend style={at={(0.97,0.7)},anchor=east,font=\small},
    ymajorgrids=true,
    grid style=dashed,
    log ticks with fixed point,
]

\addplot[
    color=blue,
    mark=triangle*,
    style=densely dashed,
    ]
    coordinates {
        (1, 28.380734387)
        (2, 355.080775134)
        (3, 489.556831309)
        (4, 436.324908652)
        (6, 443.801787742)
        (8, 340.426922416)
    };
    \addlegendentry{GMTSAR xcorr}

\addplot[
    color=red,
    mark=diamond*,
    style=densely dotted,
    ]
    coordinates {
        (1, 4.153604461)
        (2, 11.487401350)
        (3, 14.155506304)
        (4, 16.512624900)
        (6, 22.055095041)
        (8, 27.565902736)
    };
    \addlegendentry{xcorr2(单线程)}

\addplot[
    color=brown,
    mark=square*,
    style=solid,
    ]
    coordinates {
        (1, 1.850200586)
        (2, 2.137233559)
        (3, 2.451369629)
        (4, 3.073333874)
        (6, 4.204218840)
        (8, 4.960682295)
    };
    \addlegendentry{xcorr2(8线程)}

\end{semilogyaxis}
\end{tikzpicture}

        }
    \end{minipage}
}
\subfloat[CPU 核心利用率]{
    \label{fig:exp_ri_b}
    \begin{minipage}[t]{0.49\textwidth}
        \centering
        \resizebox {\textwidth} {!} {
            \begin{tikzpicture}
\begin{axis}[
    xlabel={距离插值因子},
    ylabel={占用 CPU 核心数},
    xmin=0, xmax=8,
    ymin=0, ymax=9,
    xtick={0, 2, 4, 6, 8, 10, 12},
%    ytick={ },
    legend style={at={(0.97,0.7)},anchor=east,font=\small},
    ymajorgrids=true,
    grid style=dashed,
]

\addplot[
    color=blue,
    mark=triangle*,
    style=densely dashed,
    ]
    coordinates {
        (1, 2.518)
        (2, 1.293)
        (3, 1.235)
        (4, 1.196)
        (6, 1.394)
        (8, 1.231)
    };
    \addlegendentry{GMTSAR xcorr}

\addplot[
    color=red,
    mark=diamond*,
    style=densely dotted,
    ]
    coordinates {
        (1, 1.000)
        (2, 1.000)
        (3, 1.000)
        (4, 1.000)
        (6, 1.000)
        (8, 1.000)
    };
    \addlegendentry{xcorr2(单线程)}

\addplot[
    color=brown,
    mark=square*,
    style=solid,
    ]
    coordinates {
        (1, 3.905)
        (2, 7.208)
        (3, 8.302)
        (4, 8.348)
        (6, 8.280)
        (8, 8.202)
    };
    \addlegendentry{xcorr2(8线程)}

\end{axis}
\end{tikzpicture}

        }
    \end{minipage}
}
\caption{距离向插值因子对配准速度的影响} \label{fig:exp_ri}
\end{figure}

\begin{figure}[htbp]
\centering
\subfloat[程序计算时间]{
    \label{fig:exp_ri_a}
    \begin{minipage}[t]{0.49\textwidth}
        \centering
        \resizebox {\textwidth} {!} {
            \begin{tikzpicture}
\begin{semilogyaxis}[
    xlabel={采样窗口数量},
    ylabel={计算时间/s},
    xmin=0, xmax=5000,
    ymin=0.5, ymax=250,
    ytick={0.5, 1, 2, 5, 10, 20, 50, 100, 200 },
    legend style={at={(0.97,0.16)},anchor=east,font=\small},
    ymajorgrids=true,
    grid style=dashed,
    log ticks with fixed point,
]

\addplot[
    color=blue,
    mark=triangle*,
    style=densely dashed,
    ]
    coordinates {
        (4608, 229.347642791)
        (3200, 157.700291651)
        (2048, 96.024747129)
        (1152, 51.340117550)
        (512, 21.212574205)
        (128, 5.638659853)
    };
    \addlegendentry{GMTSAR xcorr}

\addplot[
    color=red,
    mark=diamond*,
    style=densely dotted,
    ]
    coordinates {
        (4608, 25.681487185)
        (3200, 19.249105679)
        (2048, 12.470892872)
        (1152, 7.826916179)
        (512, 4.082934246)
        (128, 1.411728191)
    };
    \addlegendentry{xcorr2(单线程)}

\addplot[
    color=brown,
    mark=square*,
    style=solid,
    ]
    coordinates {
        (4608, 7.579960118)
        (3200, 5.624775143)
        (2048, 4.246431781)
        (1152, 3.118109658)
        (512, 1.859684801)
        (128, 0.914280470)
    };
    \addlegendentry{xcorr2(8线程)}

\end{semilogyaxis}
\end{tikzpicture}

        }
    \end{minipage}
}
\subfloat[CPU 核心利用率]{
    \label{fig:exp_ri_b}
    \begin{minipage}[t]{0.49\textwidth}
        \centering
        \resizebox {\textwidth} {!} {
            \begin{tikzpicture}
\begin{axis}[
    xlabel={采样窗口数量},
    ylabel={CPU 核心占用},
    xmin=0, xmax=5000,
    ymin=0, ymax=6,
    ytick={ 0, 1, 2, 3, 4, 5, 6 },
    legend style={at={(0.97,0.7)},anchor=east,font=\small},
    ymajorgrids=true,
    grid style=dashed,
]

\addplot[
    color=blue,
    mark=triangle*,
    style=densely dashed,
    ]
    coordinates {
        (4608, 2.589)
        (3200, 2.520)
        (2048, 2.439)
        (1152, 2.074)
        (512, 2.108)
        (128, 2.118)
    };
    \addlegendentry{GMTSAR xcorr}

\addplot[
    color=red,
    mark=diamond*,
    style=densely dotted,
    ]
    coordinates {
        (4608, 1.000)
        (3200, 1.000)
        (2048, 1.000)
        (1152, 1.000)
        (512, 1.000)
        (128, 0.999)
    };
    \addlegendentry{xcorr2(单线程)}

\addplot[
    color=brown,
    mark=square*,
    style=solid,
    ]
    coordinates {
        (4608, 5.899)
        (3200, 5.674)
        (2048, 5.283)
        (1152, 4.705)
        (512, 3.906)
        (128, 2.621)
    };
    \addlegendentry{xcorr2(8线程)}

\end{axis}
\end{tikzpicture}

        }
    \end{minipage}
}
\caption{采样窗口数量对配准速度的影响} \label{fig:exp_ri}
\end{figure}

\begin{figure}[htbp]
\centering
\subfloat[程序计算时间]{
    \label{fig:exp_nxy_a}
    \begin{minipage}[t]{0.49\textwidth}
        \centering
        \resizebox {\textwidth} {!} {
            \begin{tikzpicture}
\begin{loglogaxis}[
    xlabel={采样窗口数量},
    ylabel={计算时间/s},
    xmin=16, xmax=256,
    ymin=0.1, ymax=200,
    xtick={ 16, 32, 64, 128, 256 },
    ytick={0.1, 0.5, 1, 2, 5, 10, 20, 50, 100, 200 },
    legend style={at={(0.97,0.16)},anchor=east,font=\small},
    ymajorgrids=true,
    grid style=dashed,
    log ticks with fixed point,
]

% -nx NX -ny NY -norange -nointerp
\addplot[
    color=blue,
    mark=triangle*,
    style=densely dashed,
    ]
    coordinates {
        (256, 186.353415324)
        (128, 53.282931633)
        (64, 20.767185574)
        (32, 3.077585384)
        (16, 1.343715821)
    };
    \addlegendentry{GMTSAR xcorr}

\addplot[
    color=red,
    mark=diamond*,
    style=densely dotted,
    ]
    coordinates {
        (256, 47.173997387)
        (128, 12.081817861)
        (64, 4.040266771)
        (32, 1.251555321)
        (16, 0.429940367)
    };
    \addlegendentry{xcorr2(单线程)}

\addplot[
    color=brown,
    mark=square*,
    style=solid,
    ]
    coordinates {
        (256, 15.196536783)
        (128, 4.259393559)
        (64, 1.735591076)
        (32, 0.442837794)
        (16, 0.196223300)
    };
    \addlegendentry{xcorr2(8线程)}

\end{loglogaxis}
\end{tikzpicture}

        }
    \end{minipage}
}
\subfloat[CPU 核心利用率]{
    \label{fig:exp_nxy_b}
    \begin{minipage}[t]{0.49\textwidth}
        \centering
        \resizebox {\textwidth} {!} {
            \begin{tikzpicture}
\begin{semilogxaxis}[
    xlabel={采样窗口宽度},
    ylabel={CPU 核心占用},
    xmin=16, xmax=256,
    ymin=0, ymax=7.5,
    xtick={ 16, 32, 64, 128, 256 },
    ytick={ 0, 1, 2, 3, 4, 5, 6, 7 },
    log ticks with fixed point,
    legend style={at={(0.05,0.8)},anchor=west,font=\small},
    ymajorgrids=true,
    grid style=dashed,
]

\addplot[
    color=blue,
    mark=triangle*,
    style=densely dashed,
    ]
    coordinates {
        (256, 2.344)
        (128, 2.501)
        (64, 2.281)
        (32, 1.455)
        (16, 1.166)
    };
    \addlegendentry{GMTSAR xcorr}

\addplot[
    color=red,
    mark=diamond*,
    style=densely dotted,
    ]
    coordinates {
        (256, 1.000)
        (128, 1.000)
        (64, 1.000)
        (32, 0.999)
        (16, 0.998)
    };
    \addlegendentry{xcorr2(单线程)}

\addplot[
    color=brown,
    mark=square*,
    style=solid,
    ]
    coordinates {
        (256, 7.134)
        (128, 5.745)
        (64, 4.093)
        (32, 2.929)
        (16, 2.492)
    };
    \addlegendentry{xcorr2(8线程)}

\end{semilogxaxis}
\end{tikzpicture}

        }
    \end{minipage}
}
\caption{采样窗口宽度对配准速度的影响} \label{fig:exp_ri}
\end{figure}


将 GMTSAR xcorr 与 xcorr2 配准性能横向比较,xcorr2 的性能提升十分显著,部分结果性能提升甚至达到30倍之多。对于单纯的算法并行改写,这是不可能实现的。但由于 xcorr2 并非简单的并行改写而是完整的重写,并包含其他方面的程序优化(如使用实序列 FFT 算法代替复序列 FFT 算法),这样的结果是可以理解的。现代编译器能为良好的代码实现提供合适的优化,从 CPU 指令翻译的层面改善程序性能。

为了说明多线程并行的影响,可以将 xcorr2 串行性能与多线程并行性能做一个比对。串行程序严格保持了单核心 CPU 占用,说明程序本身已经能够充分发挥 CPU 单核心性能。对于三个计算参数,并行程序的加速比和 CPU 核心占用率大体上都随着数据量增加而趋于饱和(理论上最高使用率为8~9核心)。

\section{计算结果比较}

\begin{figure}[htbp]
\centering
\subfloat[GMTSAR xcorr]{
    \label{fig:exp_xcorr_result}
    \begin{minipage}[t]{0.30\textwidth}
        \centering
        \includegraphics[width=0.99\textwidth]{xcorr-result}
    \end{minipage}
}
\subfloat[xcorr2]{
    \label{fig:exp_xcorr2_result}
    \begin{minipage}[t]{0.30\textwidth}
        \centering
        \includegraphics[width=0.99\textwidth]{xcorr2-result}
    \end{minipage}
}
\subfloat[偏移矢量差值]{
    \label{fig:exp_diff_result}
    \begin{minipage}[t]{0.39\textwidth}
        \centering
        \includegraphics[width=0.99\textwidth]{diff-result}
    \end{minipage}
}
\caption{GMTSAR xcorr 与 xcorr2 配准结果对比} \label{fig:exp_result}
\note{\small 移除了从轨道数据估计的常数偏移,图中仅显示了偏移矢量长度,单位为像素。\\白色部分采样窗口最大互相关小于 GMTSAR aligh.csh 设定的最低值 20,因互相关性太差被筛除。}
\end{figure}



